\documentclass[a4paper, dvipdfmx, mathserif]{beamer}
% 印刷時には[]内に”handout”を追記する。
\usepackage{moreverb}
\usepackage{graphicx}
\usepackage{hyperref}
\usepackage{float}
\usepackage{wrapfig}
\usepackage{txfonts}
\usepackage{amsmath}
\usepackage{amssymb}
\usepackage{caption}


\bibliographystyle{junsrt}


\usetheme{default}
% Hannover, boxes, default
\usecolortheme{orchid}
\usefonttheme{default} %{professionalfonts}
\useinnertheme{rectangles} % rounded rectangles
%%%%%%%%%%%%%%%% フレーム外側のテーマの選択(省略可)
%%\useoutertheme{default}
\useoutertheme{infolines}
%% \useoutertheme{miniframes}
%% \useoutertheme{smoothbars}
% \useoutertheme{sidebar}
%% \useoutertheme{split}
%% \useoutertheme{shadow}
%% \useoutertheme{tree}
%% \useoutertheme{smoothtree}
\setbeamertemplate{navigation symbols}{}
\setbeamertemplate{footline}[frame number]
\setbeamertemplate{frametitle}[default][center]

\setbeamerfont{title}{size=\huge,series=\bfseries}
\setbeamerfont{subtitle}{size=\small,series=\bfseries}
\setbeamerfont{frametitle}{size=\large,series=\bfseries}

% \setbeamerfont{title}{size=\large}
% \setbeamerfont{frametitle}{size=\large}

% PDFのしおりが文字化けしないようにする
\usepackage{atbegshi}
\ifnum 42146=\euc"A4A2 \AtBeginShipoutFirst{\special{pdf:tounicode EUC-UCS2}}\else
\AtBeginShipoutFirst{\special{pdf:tounicode 90ms-RKSJ-UCS2}}\fi

\renewcommand{\familydefault}{\sfdefault}
\renewcommand{\kanjifamilydefault}{\gtdefault}


\newenvironment<>{varblock}[2][\textwidth]{%
\setlength{\textwidth}{#1}
\begin{actionenv}#3%
\def\insertblocktitle{#2}%
\par%
\usebeamertemplate{block begin}}
{\par%
\usebeamertemplate{block end}%
\end{actionenv}}

\captionsetup[figure]{font=scriptsize,labelfont=scriptsize}

\title[セミナー発表]{効率とシステムサイズの関係に対する確率モデルによる考察}
%\subtitle{}                 % 省略可
\author[藤本將太郎]{}
\institute{山崎研 M4 藤本將太郎}
\date{2015/05/01}

\begin{document}

%
% Title Page
%
\begin{frame}[plain]
    \titlepage
\end{frame}


%
% Outline Page
%
\begin{frame}[plain]
    \frametitle{アウトライン}
    卒業論文で扱った内容に関して
    \tableofcontents
\end{frame}


%
% Content
%
\section{研究背景}
\begin{frame}
  \frametitle{研究背景}
  \begin{itemize}
    \item 生物に関する興味
    \item アロメトリー則、ホヤの実験によるシステムサイズと代謝率の関係
    \item 会議やグループワークにおける参加人数と効率の関係
  \end{itemize}

  \begin{block}
    系のシステムサイズが大きくなると、その特徴量がシステムサイズに比例して大きくはならず、相互作用などによって期待されるより小さな量を持つ系
  \end{block}
\end{frame}

\section{作成したモデル}
\subsection{アプローチ}
\begin{frame}
\frametitle{アプローチ}
\begin{itemize}
\item 会議をイメージしながらも、より一般的。抽象的なモデルを作成
\item モデルに関して解析的な計算、数値シミュレーションを行う
\item 特徴量のシステムサイズに対する応答を調べる
\end{itemize}
\end{frame}

\section{解析結果}
\begin{frame}
    \frametitle{解析結果}
\end{frame}


\section{まとめ}
\begin{frame}
    \frametitle{テーマのまとめ}
\end{frame}




%
% Reference Page
%
\section{参考文献}
\begin{frame}
    \frametitle{参考文献}
%\beamertemplatetextbibitems
%\bibliography{reference}
\end{frame}

\end{document}
